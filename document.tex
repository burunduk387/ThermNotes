%
%Не забыть:
%--------------------------------------
%Вставить колонтитулы, поменять название на титульнике



%--------------------------------------

\documentclass[a4paper, 12pt]{article} 

%--------------------------------------
%Russian-specific packages
%--------------------------------------
%\usepackage[warn]{mathtext}
\usepackage[T2A]{fontenc}
\usepackage[utf8]{inputenc}
\usepackage[english,russian]{babel}
\usepackage[intlimits]{amsmath}
\usepackage{esint}
%--------------------------------------
%Hyphenation rules
%--------------------------------------
\usepackage{hyphenat}
\hyphenation{ма-те-ма-ти-ка вос-ста-нав-ли-вать}
%--------------------------------------
%Packages
%--------------------------------------
\usepackage{amsmath}
\usepackage{amssymb}
\usepackage{amsfonts}
\usepackage{amsthm}
\usepackage{latexsym}
\usepackage{mathtools}
\usepackage{etoolbox}%Булевые операторы
\usepackage{extsizes}%Выставление произвольного шрифта в \documentclass
\usepackage{geometry}%Разметка листа
\usepackage{indentfirst}
\usepackage{wrapfig}%Создание обтекаемых текстом объектов
\usepackage{fancyhdr}%Создание колонтитулов
\usepackage{setspace}%Настройка интерлиньяжа
\usepackage{lastpage}%Вывод номера последней страницы в документе, \lastpage
\usepackage{soul}%Изменение параметров начертания
\usepackage{hyperref}%Две строчки с настройкой гиперссылок внутри получаеммого
\usepackage[usenames,dvipsnames,svgnames,table,rgb]{xcolor}% pdf-документа
\usepackage{multicol}%Позволяет писать текст в несколько колонок
\usepackage{cite}%Работа с библиографией
\usepackage{subfigure}% Человеческая вставка нескольких картинок
\usepackage{tikz}%Рисование рисунков
\usetikzlibrary{circuits} % подключаем библиотеки, содержащие
\usetikzlibrary{circuits.ee} % УГО для схем
\usetikzlibrary{circuits.ee.IEC}
\usetikzlibrary{arrows} % подключаем библиотеки со стрелками
\usetikzlibrary{patterns} % и со штриховкой
\usepackage{float}% Возможность ставить H в положениях картинки
% Для картинок Моти
\usepackage{misccorr}
\usepackage{lscape}
\usepackage{cmap}
\usepackage{bm}
\newtheorem{definition}{Опредление}



\usepackage{graphicx,xcolor}
\graphicspath{{Pictures/}}
\DeclareGraphicsExtensions{.pdf,.png,.jpg}

%----------------------------------------
%Список окружений
%----------------------------------------
\newenvironment {theor}[2]
{\smallskip \par \textbf{#1.} \textit{#2}  \par $\blacktriangleleft$}
{\flushright{$\blacktriangleright$} \medskip \par} %лемма/теорема с доказательством
\newenvironment {proofn}
{\par $\blacktriangleleft$}
{$\blacktriangleright$ \par} %доказательство
%----------------------------------------
%Список команд
%----------------------------------------
\newcommand{\grad}
{\mathop{\mathrm{grad}}\nolimits\,} %градиент

\newcommand{\diver}
{\mathop{\mathrm{div}}\nolimits\,} %дивергенция

\newcommand{\rot}
{\ensuremath{\mathrm{rot}}\,}

\newcommand{\Def}[1]
{\underline{\textbf{#1}}} %определение

\newcommand{\RN}[1]
{\MakeUppercase{\romannumeral #1}} %римские цифры

\newcommand {\theornp}[2]
{\textbf{#1.} \textit{ #2} \par} %Написание леммы/теоремы без доказательства

\newcommand{\qrq}
{\ensuremath{\quad \Rightarrow \quad}} %Человеческий знак следствия

\newcommand{\const}{\text{const}} % Написание const в формулах

\newcommand{\qlrq}
{\ensuremath{\quad \Leftrightarrow \quad}} %Человеческий знак равносильности

\renewcommand{\phi}{\varphi} %Нормальный знак фи

\renewcommand{\epsilon}{\varepsilon}

\newcommand{\me}
{\ensuremath{\mathbb{E}}}

\newcommand{\md}
{\ensuremath{\mathbb{D}}}



%\renewcommand{\vec}{\overline}




%----------------------------------------
%Разметка листа
%----------------------------------------
\geometry{top = 3cm}
\geometry{bottom = 2cm}
\geometry{left = 1.5cm}
\geometry{right = 1.5cm}
%----------------------------------------
%Колонтитулы
%----------------------------------------
\pagestyle{fancy}%Создание колонтитулов
\fancyhead{}
%\fancyfoot{}
\fancyhead[R]{\textsc{Конспект по термодинамике и молекулярной физике}}%Вставить колонтитул сюда
%----------------------------------------
%Интерлиньяж (расстояния между строчками)
%----------------------------------------
%\onehalfspacing -- интерлиньяж 1.5
%\doublespacing -- интерлиньяж 2
%----------------------------------------
%Настройка гиперссылок
%----------------------------------------
\hypersetup{				% Гиперссылки
	unicode=true,           % русские буквы в раздела PDF
	pdftitle={Заголовок},   % Заголовок
	pdfauthor={Автор},      % Автор
	pdfsubject={Тема},      % Тема
	pdfcreator={Создатель}, % Создатель
	pdfproducer={Производитель}, % Производитель
	pdfkeywords={keyword1} {key2} {key3}, % Ключевые слова
	colorlinks=true,       	% false: ссылки в рамках; true: цветные ссылки
	linkcolor=blue,          % внутренние ссылки
	citecolor=blue,        % на библиографию
	filecolor=magenta,      % на файлы
	urlcolor=blue           % на URL
}
%----------------------------------------
%Работа с библиографией 
%----------------------------------------
\renewcommand{\refname}{Список литературы}%Изменение названия списка литературы для article
%\renewcommand{\bibname}{Список литературы}%Изменение названия списка литературы для book и report
%----------------------------------------
\begin{document}
	\begin{titlepage}
		\begin{center}
			$$$$
			$$$$
			$$$$
			$$$$
			{\Large{НАЦИОНАЛЬНЫЙ ИССЛЕДОВАТЕЛЬСКИЙ УНИВЕРСИТЕТ}}\\
			\vspace{0.1cm}
			{\Large{ВЫСШАЯ ШКОЛА ЭКОНОМИКИ}}\\
			\vspace{0.25cm}
			{\large{Факультет физики}}\\
			\vspace{5.5cm}
			{\Huge\textbf{{Конспект}}}\\%Общее название
			\vspace{1cm}
			{\LARGE{<<Термодинамика и молекулярная физика - временно сырая>>}}\\%Точное название
			\vspace{1cm}
			{\LARGE{Автор курса: Кунцевич А. Ю.}}\\%Лектор
			\vspace{2cm}
			\vfill
			\includegraphics[width = 0.2\textwidth]{HSElogo}\\
			\vfill
			Москва\\
			2020
		\end{center}
	\end{titlepage}
	
	\tableofcontents
	\newpage
	\addcontentsline{toc}{section}{Замечания и благодарности}
	\section*{Замечания и благодарности}
	Данный конспект написан студентами и для студентов. Он может содержать опечатки, неточности и серьёзные смысловые ошибки.
	Над ним работали:
	\begin{itemize}
		\item Написание:
		\subitem М. Блуменау 
		\item Шаблон для Latex:
		\subitem С. Захаров и М. Парфёнов
		\item Поиск ошибок и опечаток:
		\subitem М. Марков
		\subitem А. Эбериль
		\subitem А. Никонов
	\end{itemize}
	\href{https://drive.google.com/drive/folders/1YxoCMlbHhKd1kKbX7nwRKx5jfchXqmVf?usp=sharing}{Ссылка на записи лекций} 
	\newline
	\href{https://github.com/burunduk387/HSE-FF/tree/main/Thermodynamics}{Ссылка на репозиторий с исходными файлами} 
	\newpage
	\addcontentsline{toc}{section}{Введение}
	\section*{Введение}
	Курс читается на Факультете Физики НИУ ВШЭ в 2020 году впервые. Традиционный курс Физтеха достаточно плохо усваивался студентами в силу ряда причин, что  и натолкнуло на его изменение.
	
	О методах контроля:
	\begin{itemize}
		\item Каждую неделю присылается домашнее задание (набор из нескольких задач) на две недели.
		\item Будет проведена контрольная и коллоквиум в формате вопроса по выбору.
		\item Накопленная оценка складывается следующим образом: 0,35*Контрольная + 0,25*Коллоквиум + 0,4*Домашние задания.
		\item Итоговая оценка: 0,6*Накопленная + 0,4*Экзамен.
		\item Автоматы выставляются студентам с накопленными оценками 8, 9 и 10.
	\end{itemize}
	\newpage
	\section{Лекция 1}
	\subsection{Основные термины}
	Курс содержит три части, каждой из которых можно дать краткое определение:
	\begin{itemize}
		\item Термодинамика - наука о закономерностях для усреднённых величин, описывающих макросистемы.
		\item Статистическая физика - наука о том, как устройство системы на микро-уровне определяет усредненные параметры макросистемы.
		\item Кинетика - наука, изучающая отклик макросистемы на внешнее воздействие.
	\end{itemize}
	Приведем пару примеров:
	
	Пример 1: Газ вблизи перехода в жидкое состояние.
	
	Термодинамика: теплоемкость, фазовая диаграмма P(T).
	
	Стат. Физика: как термодинамические величины следуют из свойств молекул.
	
	Кинетика: вязкость, теплопроводность, скорость зародышеобразования в переохлажденном газе.
	
	Пример 2:  Ферромагнетик вблизи точки Кюри.
	
	Термодинамика: описание фазового перехода 2 рода – восприимчивость и теплоемкость.
	
	Стат. Физика: вывод свойств  перехода из микроскопических соображений, модели Изинга, Гейзинберга, среднего  поля.
	
	Кинетика: спиновые волны, движение доменных стенок, время релаксации спина.
	
	В дальнейшем также понадобятся следующие определения (здесь - упрощённые, полные можно найти, например в «Теоретической физике» Л. Д. Ландау и Е. М. Лившица):
	\begin{itemize}
		\item Микросостояние – все значения полного набора переменных описывающих систему. 
		\newline
		(например: идеальный газ – значения координат и скоростей всех молекул; магнетик – значения спина всех частиц.)
		\item Макросостояние – последовательность микросостояний системы, характеризующаяся средними значениями физических величин (например: давление, объём, средняя энерги запасённая в системе, флуктуация скоростей...).
		\item Состояние равновесия – макросостояние, в котором все средние значения стабильны.
		\item Релаксация (термализация)– процесс установления равновесия.
		\item Время релаксации – время, по прошествии которого можно говорить о равновесии. 
	\end{itemize}
	
	Однако, существуют системы, которые никогда не приходят в положение равновесия, например стёкла. Так, у них нет температуры плавления и термализованы не все степени свободы.
	
	Также приведем статистическое определение слова «Температура», пусть и несколько ошибочное (правильное - 5 том «Теоретической физики»):
	\newline
	Система, пришедшая в состояние теплового равновесия в своем макросостоянии реализует последовательность микросостояний, вероятность которых подчиняется распределению Гиббса: 
	\begin{equation*}
		P_i=P_0 e^{-E_i/kT}
	\end{equation*}
	
	Что это такое? Представим газ в комнате. Нужно взять микросостояние, описанное всеми координатами и всеми скоростями молекул газа, и посчитать полную энергию. Вероятность этого состояния пропорциональна 
	\begin{equation*}
		e^{-E_i/kT} 
	\end{equation*}
	($P_0$ играет роль коэффициента нормировки). 
	
	До сих пор происходят споры по поводу справедливости верхней формулы, так как в термодинамических системах в основном реализуются состояния вблизи наиболее вероятных. 
	
	Есть ли вероятность того, что какая-то из молекул приобретёт релятивистскую скорость?
	Строго говоря, у этого события есть какая-то энергия и вероятность. Но фактически это событие никогда не реализуется. Это всё равно, что будет печатать группа обезьян, и они наберут все собрания сочинений Льва Толстого. Причём в случае с молекулой с околосветовой скоростью это можно строго доказать. Если она возникла в комнате, то она должна была родиться ещё откуда-то. Тогда, «вернувшись в прошлое» у нас будет 2 молекулы с примерно половинными скоростями. Отмотав время ещё сильнее вспять, получим 4 молекулы с большими скоростями и так далее. Понятно, что динамически представить такую ситуацию очень сложно и её вероятность 0. 
	
	В жизни несколько проще: у нас есть термометр. Это некоторый измерительный прибор. Тем не менее, нельзя сказать следующее: «Температура - это то, что измеряет термометр», поскольку два термометра разных производителей могут показывать разную «температуру». Первичным источником знания о температуре можно считать идеальный газ. Есть метрологи, доказывающие, что тот или иной прибор действительно измеряет температуру.
	
	Что называется средней величиной?
	Средним значением величины A по распределению называется 
	\begin{equation*}
		\frac{\sum_{i} A_i P_i }{\sum_{i} P_i},
	\end{equation*}
	где сумма берется по всем микросостояниям системы.
	Измеряемые величины, как правило, являются средними. 
	
	Например: средняя энергия, средняя скорость, средняя флуктуация энергии в заданном объёме, среднее значение концентрации частиц в заданном объеме, среднее значение спина. 
	Или выражаются через них, например давление, теплоёмкость.
	
	Существуют дискретные и непрерывные величины. Примерами вторых могут быть: импульс, координаты. Дискретные же, например, встречаются в квантовой механике.
	\subsection{Распределение Больцмана для 1 молекулы}
	Решим задачу: Распределение Больцмана для 1 молекулы газа. Найти положение h её центра тяжести. Температура везде одинаковая и равна T. Высоту считать дискретной величиной.
	
	Решение:
	\begin{equation*}
		\begin{aligned}
			& E=E_{k} + mgH_{i}                                                                               \\
			& P(h)=P_{0} e^{\frac{E_{k}+mgH}{kT}}                                                             \\
			& H= hi                                                                                           \\
			& \tilde{H} = \frac{\sum_{i} h i e^{\frac{-mghi+E_{k}}{kT}}}{\sum_{i} e^{\frac{-mghi+E_{k}}{kT}}} \\
		\end{aligned}
	\end{equation*}
	Поскольку кинетическая энергия от высоты никак не зависит, числитель и знаменатель можно сократить на сумму членов с ней. Далее заметим, что нижний ряд - геометрическая прогрессия, а верхний ряд получается из нижнего дифференцированием по параметру:
	\begin{equation*}
		\begin{aligned}
			& \tilde{H} = \frac{\sum_{i} hi e^{\frac{-mghi}{kT}}}{\sum_{i} e^{\frac{-mghi}{kT}}} \\
			& I = \sum_{i} e^{\frac{-mghi}{kT}} = \frac{1}{1-e^{\frac{-mgH}{kT}} }               \\
			& \alpha =\frac{mgh}{kT}                                                             \\
			& \frac{dI}{d \alpha} = h \frac{e^{- \alpha}}{(1-e^{- \alpha})^2}                    \\
			& \tilde{H} = \frac{e^{- \alpha}}{1-e^{- \alpha}}                                    
		\end{aligned}
	\end{equation*}
	Мы можем свести высоту к непрерывной степени свободы, устремив h к нулю, тогда:
	\begin{equation*}
		\begin{aligned}
			& e^{- \alpha} \approx = 1 - \alpha                                                        \\
			& \tilde{H} = \frac{e^{- \alpha}}{1-e^{- \alpha}} \approx \frac{h}{\alpha} = \frac{kT}{mg} 
		\end{aligned}
	\end{equation*}
	Тот же результат можно получить, перейдя к интегралу и считая высоту непрерывной величиной. Также, как видно из формулы, чем выше температура, тем выше будет и центр тяжести частицы.
	
	Это можно понять, подумав о том, что сила тяжести будет давить на частицу, или же, попробовав подставить разные предельные случаи температуры в изначальный ряд. Можно воспользоваться "моделью экспоненты простого человека", которую надо представлять так: сначала она ноль, потом перескакивает на 1, и в какой-то из точек становится бесконечностью (как бы ступенька и потом прямой угол).
	
	Таким образом, непрерывные степени свободы, при желании, можно сводить к дискретным. Только что мы свели непрерывную высоту к дикретной и всё равно правильно решили задачу.
	
	\subsection{Идеальный газ как система независимых подсистем}
	Идеальный газ - система не взаимодействующих друг с другом частиц.
	
	Обычно указывается ещё и нулевой объем, но поскольку они не взаимодействуют, это роли не играет. На языке вероятности:
	
	P(частица 1 в сост. А и частица 2в сост. В) = P(частица 1 в сост. А) * P(частица 2 в сост. В)
	
	На языке энергий: 
	\begin{equation*}
		E = E_{1} + E_{2}
	\end{equation*}
	
	На языке распределения Гиббса:
	\begin{equation*}
		P(1 in A , 2 in B)  \propto{ e^{- \frac{E_{1}}{kT}}  e^{- \frac{E_{2}}{kT}}} =  e^{- \frac{E}{kT}}
	\end{equation*}
	
	Каждая частица имеет своё распределение, при этом они "не чувствуют" друг друга. Энергии складываются, вероятности перемножаются.
	
	Однако, есть несколько подводных камней. В частности, вспомним нашу релятивистскую частицу. Допустим, что она у нас существует. Тогда, каким же образом она придет в равновесие? Ведь частицы в идеальном газе не взаимодейтсвуют. Никакого термодинамического равновесия не установится. Для термодинамического равновесия нужно взаимодействие частиц. 
	\subsection{Вопросы усреднения и эргодические системы}
	У многих мог возникнуть вопрос: а по какой переменной берётся среднее? Неявно - по времени. Ещё можно взять среднее по ансамблю (частицам). 
	
	Например, какова средняя скорость молекул в комнате? Можно взять 1 молекулу и следить за ней (по показаниям секундомера/спидоматра и определять скорость). А можно взять 1000 молекул, "сфотографировать их", и поскольку это все вероятные состояния, между которыми переключается система (то есть мы видим через какие состояния пройдёт каждая из молекул) и из этого найти скорость. Когда усреднение по ансамблю и времени даёт один и тот же ответ, мы видим эргодичную систему. 
	
	Примером неэргодической системы может быть следующее:
	Алексанндр Юрьевич открыл казино. Оставим в стороне вопросы, связанные с законом, и рассмотрим его казино поближе. Правила такие: играть можно только на все деньги (опять же, тут, вероятно, поможет гадалка). Подбрасывается идеальная монета (шанс выпадения орла равен шансу выпадения решки и оба равны 50 процентам). Выпадает орёл и сумма увеличивается в 1,5 раза, выпадает решка - умножается на 0,5.
	
	Теперь в это популярное казино пришёл ансамбль. У каждого музыканта было ровно 100 долларов. Тогда, сыграв разок, половина из них может уйти со 150, другие унесут лишь 50. Казалось бы, убыточное заведение (ведь усреднение по ансамблю показывает равенство притока денег от музыкантов и оттока). Но теперь Александр Юрьевич подходит к каждому из них и предлагает сыграть. Тогда пусть они сыграли N раз (N - чётное). В таком случае музыкант выйдет с $1,5^{\frac{N}{2}} 0,5^{\frac{N}{2}} \cdot 100 = (1+0,5)^{\frac{N}{2}} (1-0,5)^{\frac{N}{2}} \cdot 100= (1^{2}-(0,5)^{2})^{\frac{N}{2}} \cdot 100$, что явно меньше 100. Сыграв достаточно раз, бедный музыкант разорится совсем. 
	
	Как видно, усреднение по времени и по ансамблю дало разный результат, что и является примером неэргодической системы. Ещё один пример - стёкла. Остальные же, в основном, эргодические. 
	\subsection{Двухуровневые системы}
	Теперь пару слов о двухуровненевых системах. В принципе, их можно считать той же идеальной монетой, только квантовой, но это не слишком усложняет. Пусть в атоме есть электрон (а они там есть). Как известно, электроны занимают дискретные уровни. Если мы будет смотреть на атом со стороны распределения Гиббса, мы должны сказать, что уровни посчитаны относиттельно центра масс, а он сам имеет непрерывную степень свободы. С помощью ранее указанных формул можно посчитать распределение. Нужно понимать, что в атоме электрон предпочитает находиться на нижнем уровне. Почему? Потому что мы живем при комнатной температуре, а потенциальный барьер, который нужно преодолеть находится примерно на 100 000К. Из интересного можно отметить, что если энергия в атоме водорода станет очень большой, то электрон сможет и оторваться. 
	
	Рассмотрим искусственный объект - квантотвая точка. В ней тоже есть уровни. Они бывают вморожены в полупроводник, тогда у неё есть только дискретная степень свободы.
	
	Тем не менее, самое удобное представление подобных систем для физиков - спин. Спин рисуют как стрелочку, но его состояние можно представить так: он может смотреть либо вверх, либо вниз, а также комбинация этих двух. Замечателен он тем, что им можно управлять через его магнитный момент. То есть, если прикладывать в нему магнитное поле, можно увидеть, что энергия спина, направленного по полю, становится меньше, а направленного против - возрастает. Одиночный спин - вполне реализуемый объект: такой двухуровневой системой можно управлять. То есть у Вас есть "монета", а Вы, управляя внешним полем можете изменять энергию спина вверх и спина вниз. Из этого удобно рассмотреть отрицательную, нулевую и бесконечную температуру. Ещё поле в 1Т расщепляет спин приблизительно на 1К (в случае электрона, ядро в 1000 раз больше).
	\subsection{Стекло}
	Небольшой рассказ про стёкла. Есть атомы, которые стоят в кристаллической решётке. Обычно, атомы расположены так, чтобы минимизировать свою свободную энергию. Теперь рядом с этой решёткой стоят атомы другой решетки, но обладающие совершенно иной ориентацией. Такое состояние нестабильно. То есть атом из одной решетки может встать в другое место. Тогда из-за такого действия мы выиграем энергию в одном кусочке материала, но проиграем в другом. Фактически атому нужно преодолеть барьер, чтобы перейти в другое состояние. Такое просто так произойти не может. В стекле же рядом могут стоять решетки с подобной ориентацией, а тогда частице надо гораздо меньше энергии и из-за этого оно течет. То есть стекло тело твёрдое, но кристаллом не является. Допустим мы из "стекла" (кварца) хотим получить кристалл. Для этого его надо расплавить, потом дождаться начала роста кристалла и очень медленно его охлаждать.
	
	\section{Лекция 2}
	\subsection{Частица в трёхмерном пространстве}
	Предположения:
	\begin{itemize}
		\item $E = \frac{p_{x}^2}{2m}+\frac{p_{y}^2}{2m}+\frac{p_{z}^2}{2m}$
		\item Фазовое пространство заполняется однородно по импульсу (свойство однородности пространства и сопряжённости импульса и координаты (доказывается в квантовой механике))
		\item  Распределение Гиббса $P = P_{0} e^{\frac{-E}{kT}}$
	\end{itemize}
	
	Рассматриваем частные случаи. Приведём случай, когда не учитывается гравитация. В одномерном случае вероятность найти частицу с импульсом, лежащем в промежутке:
	\begin{equation*}
		P(p_{1x} < p < p_{x2}) = \frac{\int_{p_{x1}}^{p_{x2}} e^{\frac{-p^{2}} {2mkT} }dp}{ \int_{-\infty}^{+\infty} e^{\frac{-p^{2} } {2mkT} }} 
	\end{equation*}
	Нижний интеграл называется Гауссовым:
	\begin{equation*}
		\int_{-\infty}^{+\infty} e^{-ax^{2}} = \sqrt{\frac{\pi}{a}}, \; \text{тогда:} \; P(p_{1x} < p < p_{x2}) = \frac{\int_{p_{x1}}^{p_{x2}} e^{\frac{-p^{2}} {2mkT} }dp}{ \sqrt{2\pi m k T} }
	\end{equation*}
	
	Теперь рассмотрим 2D случай. Тут мы уже говорим о вероятности найти частицу в области фазового пространства $S_{p}$, всё фазовое пространство обозначим S:
	\begin{equation*}
		P(S) = \frac{\int_{S_{p}} e^{\frac{-p_{x}^{2} -p_{y}^{2}} {2mkT}} dp_{x} dp_{y}}{ \int_{S} e^{\frac{-p_{x}^{2} -p_{y}^{2}} {2mkT} }} = \frac{\int_{S_{p}} e^{\frac{-p_{x}^{2} -p_{y}^{2}} {2mkT}} dp_{x} dp_{y}}{ 2\pi m k T}
	\end{equation*} 
	Т. к.  нижний интеграл берётся по двум независящим друг от друга величинам, он равен произведению уже известных нам (см. предыдущий пункт про 1D).
	Верхний же интеграл в большинстве практических случаев симметрчен относительно начала координат. В сферически симметричном случае:
	\begin{equation*}
		\begin{aligned}
			& p_{x}^2+p_{y}^2=p^2                                                                                         \\
			& P(S)=\frac{ \int 2 \pi p e^\frac{-p^2}{2mkT} dp}{2 \pi mkT} = \int e^{-\xi} d \xi, \xi = \frac{p^{2}}{2mkT} 
		\end{aligned}
	\end{equation*}
	По аналогии и 3D случай:
	\begin{equation*}
		P(S) = \frac{\int_{S_{p}} e^{\frac{-p_{x}^{2} -p_{y}^{2}+-p_{z}^{2}} {2mkT}} dp_{x} dp_{y} dp_{z}}{ (2 \pi mkT)^{3/2}} = \frac{\int 4 \pi p^{2}  e^{\frac{-p^{2}}{2mkT}}dp}{ (2 \pi mkT)^{3/2}}
	\end{equation*}
	Полученное выражение назвают распределением Максвелла. Поскольку физика наука экспериментальная, следует думать об измеряемых величинах, которые являются средними. Рассмотрим среднюю скорость для трёхмерного случая: 
	\begin{equation*}
		\begin{aligned}
			& \overline{v} = \frac{1}{ (2 \pi mkT)^{3/2}} \int_{0}^{\infty} \frac{p}{m} 4 \pi p^{2}  e^{\frac{-p^{2}}{2mkT}}dp = |\xi  =  \frac{p^{2}}{2mkT}| = \frac{4kT}{(2 \pi mkT)^{1/2}} \int_{0}^{\infty} \xi e^{-\xi}  = &   \\
			= \frac{4kT}{(2 \pi mkT)^{1/2}}
		\end{aligned}
	\end{equation*}
	Вычислим теперь среднеквадратичную скорость и убедимся, что она отличается (сначала, для удобства, вычислим просто средний квадрат скорости):
	\begin{equation*}
		\overline{v^{2}} = \frac{1}{ (2 \pi mkT)^{3/2}} \int_{0}^{\infty} \frac{p^{2}}{m^{2}} 4 \pi p^{2}  e^{\frac{-p^{2}}{2mkT}}dp 
	\end{equation*}
	Воспользуемся фактами: 
	\begin{equation*}
		\begin{aligned}
			& \frac{d^{n}}{d \alpha^{n}} \int_{0}^{\infty} e^{-\alpha x^{2}} dx = \int_{0}^{\infty} x^{2n} e^{-\alpha x^{2}} = \frac{1}{2} \frac{d^{n}}{d\alpha^{n}}\sqrt{\frac{\pi}{a}} \\
			& \overline{v^{2}} = \frac{4\pi}{m^{2} (2\pi mkT)^{3/2}} \frac{3 \sqrt{\pi} (2mkT)^{5/2}} {8} = \frac{3kT}{m}                                                                
		\end{aligned}
	\end{equation*}
	Тогда:
	\begin{equation*}
		\sqrt{\overline{v^{2}}} =   \frac{(3kT)^{1/2}}{m^{1/2}}
	\end{equation*}
	График распределения вероятности некой функции по Максвеллу принимает вид "колокльчика".
	Взяв производную от подынтегральной функции можно получить наиболее веротяную скорость, равную: $v_{mp} = \frac{()2kT)^{1/2}}{m^{1/2}}$
	Рассмотрим более простой способ получения среднего квадрата скорости:
	\begin{equation*}
		\overline{v^{2}} = \frac{\int_{-\infty}^{\infty} dp_{x}  \int_{-\infty}^{\infty} dp_{y} \int_{-\infty}^{\infty} dp_{z} e^{- \frac{p_{x}^2+p_{y}^2+p_{z}^{2}}{2mkT}} (\frac{p_{x}^2+p_{y}^2+p_{z}^{2}}{m^2}) }{(2 \pi  m k T )^{1/2}}
	\end{equation*}
	Поскольку все импульсы равноправны, можно взять лишь один интеграл и умножить его на три. Тогда: 
	\begin{equation*}
		\overline{v^{2}} =   \frac{3kT}{m}
	\end{equation*}
	\subsection{Давление}
	Теперь решим две задачи: есть идеальный газ и площадка площадью $S$. Сначала посчитаем число молекул, ударяющихся о неё. Пусть время наблюдения равно $\tau$. Если площадка достаточно большая, нам важно лишь перпендикулярное движение частичек. Частица, летящая к стенке, имеет скорость $v_{x}$ и импульс $p_{x}$. За время наблюдения в стенку врежутся частицы, находящиеся на $l_{x} = \frac{p_{x}}{m} \tau$. Объем равен $V=l_{x} S$. Понятно, что число частиц содержит все частицы, нас же интересует только та часть, которая имеет подходящий по направлению импульс: $dN   = nV \frac{ dp_{x} e^{-\frac{p_{x}^2}{2mkT}}}{(2 \pi mkT)^{1/2}}$, где $n$ - концентрация, а $\frac{e^{-\frac{p_{x}^2}{2mkT}}}{(2 \pi mkT)^{1/2}}$ - вероятность.
	Подставим все величины в $dN$ и подсчитаем полное число частиц:
	\begin{equation*}
		N  = \int_0^{\infty}   \frac{ \tau p_{x} S dp_{x} e^{-\frac{p_{x}^2}{2mkT}}}{(2 \pi mkT)^{1/2} m } = \frac{\tau n S (kT)^{1/2}}{(2 \pi m )^{1/2}} = m n S \tau \frac{\overline{v}}{4}
	\end{equation*}
	Теперь поситаем давление:
	\begin{equation*}
		p = P \tau S  = nkTS \int_{0}^{\infty}  \frac{ \tau 2 p_{x}^{2} dp_{x} e^{-\frac{p_{x}^2}{2mkT}}}{(2 \pi mkT)^{1/2} }
	\end{equation*}
	Данный интеграл очевидно берется и получается всем известный ответ:
	\begin{equation*}
		P = nkT
	\end{equation*}
	\subsection{Вращательное движение молекул}
	Небольшое отступление: Есть вращательное движение молекул. Пусть конструкция представляет себя гантельку. Молекулы в ней - квантовые сущности. А поскольку при вращении вдоль оси самой гантели она переходит сама в себя, данной степени свободы у неё нет. Поэтому когда считается её энергия, нужно сделать следующее: $E = E_{K} + E_{p}$, где для вычисления потенциальной энергии вращения его можно задать двумя углами. Тогда, условно скажем, $E = \frac{p_{cm}}{2m} + \frac{L^2}{2I}$.  Это значит, что когда считается полная энергия, нужно интегрировать и по моменту импульса, что приводит к возникновению $5/2$. Трёхатомная молекула задатся уже тремя углами (как и твердое тело), из-за чего $E = 3kT$, (возникает $6/2$).
	\subsection{Распределение средних величин}
	Теперь несколько слов про распределение средних величин. Предположим, что мы бросаем обычный шестиранный кубик, вероятность выпадения любой из граней одинакова. Среднее значение равно 3.5. Бросили кубик два раза. Исходы такие: 2 3 4 5 6 7 8 9 10 11 12. У 2 шанс $1/36$ (как и у 12), у 3 - $1/18$ (посчитайте исходы каждого из двух бросков и убедитесь в этом) и так далее. Чем больше мы будем бросать кубик, тем больше распределение будет напоминать распределение Гаусса (нормальное). 
	\section{Лекция 3}
	\subsection{Среднеквадратичное отклонение}
	Среднеквадратическое отклонение определяется так:
	\begin{equation*}
		(A - \overline{A})^2
	\end{equation*}
	Чтобы посчитать его, надо:
	\begin{equation*}
		\frac{\sum_{i} (A_{i} - \overline{A})^2  p_{i} }{\sum_{i} p_{i}}
	\end{equation*}
	Когда события нескоррелированы, их вероятности перемножаются:
	\begin{equation*}
		\begin{aligned}
			& \overline{A_{i} B_{k}} = \overline{A} \overline{B}     \\
			& \overline{A_{i} + B_{k}} = \overline{A} + \overline{B} \\
		\end{aligned}
	\end{equation*}
	Раскрыв скобки во втором выражении получим:
	\begin{equation*}
		\overline{A^{2}} - \overline{A}^{2}
	\end{equation*}
	\subsection{Примеры}
	Теперь рассмотрим пример со спином (в нулевом магнитном поле) $\pm 1/2$
	Вероятность каждого из состояний равна, посчитаем среднеквадратическое отклонение:
	\begin{equation*}
		\overline{S^{2}} - \overline{S}^{2} = 	\frac{1}{4} - 0 = 	\frac{1}{4} 
	\end{equation*}
	Посмотрев на столь простой пример, можно посмотреть на распределённую по Максвеллу частицу. Её средняя энергия:
	\begin{equation*}
		\overline{E} = \frac{3kT}{2}
	\end{equation*}
	Посчитаем теперь среднее квадрата (вспомнив трюк с дифференцированием по параметру):
	\begin{equation*}
		\overline{E^{2}} = \frac{1}{(2 \pi mkT)^3/2} \int_0^{\infty} 4 \pi p^{2} e^{-\frac{p^2}{2mkT}} (p^{2})^{2} dp = \frac{15 (kT)^{2}}{4}
	\end{equation*}
	Среднеквадратическое отклонение:
	\begin{equation*}
		\sqrt{\overline{E^{2}} - \overline{E}^{2} } = kT \sqrt{\frac{15}{4} - {9}{4}} = \sqrt{\frac{3}{2}} kT
	\end{equation*}
	Пусть у нас есть $B = A1 + A2 + ... + AN$. Нужно найти среднеквадратическое отклонение этой величины. Предположим, что средние величины и дисперсии этих случайных событий равны. Тогда:
	\begin{equation*}
		\begin{aligned}
			& \overline{B} = N \overline{A}                                                                                                                                             \\
			& \overline{B^{2}} = \overline{A1^{2} }+ \overline{A2^{2}} + ... + \overline{AN^{2}} + \overline{A1A2} + \overline{A1AN} + \overline{A2A1} + \overline{ANA1} + ... =        \\
			& =  \overline{A1^{2} }+ \overline{A2^{2}} + ... + \overline{AN^{2}} +N(N-1)\overline{A}^{2} = N  \overline{A^{2} } +N(N-1)\overline{A}^{2}                                 \\
			& \sqrt{ \overline{B^2} - \overline{B}^2 } =\sqrt{N  \overline{A^{2} } +N(N-1)\overline{A}^{2} - N^{2}\overline{A}^{2}  } = \sqrt {N (\overline{A^{2} }- \overline{A}^{2})} \\
		\end{aligned}
	\end{equation*}
	\subsection{Флуктуация}
	То есть если у нас сама частица имела среднеквадратическое отклонение, а мы взяли $N$ частиц и поделили среднеквадратическое отклонение на $N \overline{A}$, мы получим относительную флуктуацию. Тогда станет очевидно, что в большом ансамбле меньше труднее возникнуть аномалиям.
	Теперь посчитаем среднеквадратическую флуктуацию числа частиц. Пусть у нас есть большой объём. Выберем в нём маленький объёмчик $V$. Вероятность найти каждую частицу в нём равна $p = \frac{V}{V_{all}}$. Среднее количество частиц очевидно и равно $\overline{N} = N_{all} p$. Средний квадрат одной частицы: $\overline{n^{2}} = p \cdot 1^{2} + (1-p) \cdot 0^{2}$; $\overline{n^{2}} - \overline{n}^{2} = p - p^{2}$. Тогда по только что выведенной формуле: $\overline{N^{2}} - \overline{N}^{2} = N(p - p^{2})$. Но объемчик был очень маленьким, поэтому можно пренебречь $p^{2}$. Тогда среднеквадратическое отклонение равно $\sqrt{\overline{N}}$.
	Бывает удобно пользоваться теоремой распределения по степеням свободы: $\overline{E} = ... \overline{\alpha x^{2}} ...$, где $ \overline{\alpha x^{2}} = \frac{kT}{2}$.
	
	Теперь возьмем такой прибор как динамометр. Как можно догадаться, в энергии будет член, подобный вышесказанному. Благодаря нему и будет возникать маленький дребезг, не позволяющий измерять малые величины. 
	\subsection{Шум резистора}
	Рассмотрим резистор. Каждый резистор, как оказалось, шумит в силу флуктуации электронов. Конденсаторы и индуктивности (идеальные) не шумят. Чтобы посчитать этот шум нужно вспомнить электричество и магнетизм, а именно длинные линии. По ней может распространяться сигнал. Делает это он со скоростью света. Существует понятие волнового сопротивления, размерность которого $\sqrt{\frac{L}{C}}$. То есть можно представить себе длинную линию как множество соединённых конденсаторов и катушек. Волна по линии идет бесконечно, но на краю волна подчинится волновому уравнению (возникнут граничные условия). Если на конце сопротивление, равное волновому, то волна уйдёт в него. Представим, что у нас есть резистор, который шумит. Пусть у нас есть источник напряжения, соединённый с ним последовательно. В случае источника тока соединение будет параллельное. Источники при этом посчитаем идеальными. Добавим такой же источник и резистор с другого конца и два ключа. Линии у нас согласованные. Сигнал идет с одного источника, приходит на резистор с другой стороны и рассеивается там в виде тепла. Состояние равновесное, вечного двигателя не случается. Так сигналы и перекрестно идут в системе. В соответствии с правилами Кирхгофа, половина напряжения падает на первом резисторе, половина на втором (сопротивления у них равны). Проведем мысленный эксперимент. Количество энергии, которое в систему приходит от резистора равно $\frac{V_{noise1}^2 l} {4R_{noise1} c }$. Теперь замкнём ключи. На резисторы ток не пойдет, но энергия в системе осталась. это приводит к тому, что длинная линия становится резонатором. В нём есть граничные условия, в частности, они означают, что напряжение на концах равно 0. Запасённая энергия разложится по волнам. При этом $U = U_{i} sin(\pi x/ l  + \omega_{i} t) $.  Но на каждую из волн энергия пропорциональна квадрату напряжения этой волны и равна $kT/2$. $\omega_i = \frac{\pi i c t}{L}$.  Напряжение шума тоже можно разложить по гармоникам и тогда получится: $\frac{LV_{N1i}^{2}}{4Rc}=\frac{kT}{2}$ - то, сколько энергии дал резистор на некоторой гармонике. То есть, она приходится на интервал частот $\Delta f= \frac{c}{2L}$. В итоге получаем:
	\begin{equation*}
		V_{i}^{2}=\frac{4kTR}{\Delta f}
	\end{equation*}
	\section{Лекция 4}
	\subsection{Энтропия}
	На прошлой лекции мы видели, что при большом числе частичек, ширина их распределения на самом деле узкая. Средние величины макросостояния разбросаны очень узко. Закон распределения - нормальный. Теперь представим, что энергия - дискретная величина и рассмотрим самое вероятное состояние. Пусть число способов его реализации равно Г. Назовём величину 
	\begin{equation*}
		S = k ln(\Gamma(\overline{E}))
	\end{equation*}
	энтропией. Рассмотрим систему, состоящую из двух подсистем, которые не взаимодействуют (если являются достаточно большими). У каждой из них есть собственная энергия, а общая энергия равна их сумме (в силу аддитивной энергии). Также у каждой из них есть и Г. Каким же количеством способов реализуется большая система? Перемножением способов каждой из подсистем. Для энтропии:
	\begin{equation*}
		S = kln(\Gamma) =  kln(\Gamma_{1} \cdot \Gamma_{2}) = k(ln(\Gamma_{1}) + ln(\Gamma_{2}))
	\end{equation*}
	То есть энтропия складыается (иными словами - аддитивна). Энтропия введена не просто так.  Вероятность найти частицу в состоянии энергии пропорционально:
	\begin{equation*}
		P_{E} \propto \Gamma_{E} e^{-E/kT} = e^{-E/kt + ln(\Gamma)} = e^{-\frac{E - TS}{kT}}
	\end{equation*}
	\subsection{Свободная энергия}
	Теперь введём свободную энергию. Ею называют величину $F =E-TS$. Очевидно, что наиболее вероятному состоянию системы соответствует минимум свободной энергии (достаточно посмотреть на показатель экспоненты). Вспомним механику: Сила действовала так, чтобы уменьшить просто энергию. В термодинамике же добавляется множитель Г. То есть система стремится минимизировать энергию и максимизировать энтропию. Чтобы это перестало казаться чем-то искусственным, рассмотрим в качестве модели набор двухуровневых систем. Первый уровень будет 0, а второй - неким $\Delta E$.  Каковым может быть состояние данной системы? В целом, практически любым. Назовем состояние на 1 уровне основным, а на верхнем - возбужденным. Число систем N. Пусть m из них возбужденных. Полная энергия $E = m \Delta E$. Посчитаем теперь число способов:
	\begin{equation*}
		\begin{aligned}
			& \Gamma = C_{N}^{m} = \frac{N!}{m!(N-m)!}                           \\
			& ln(\Gamma) = \frac{1}{2 }(ln(\frac{N}{n(N-m)})- ln 2 \pi) + NlnN - \\
			& - mlnm - (N-m)ln(N-m) = N( -plnp - (1-p)ln(1-p))                   
		\end{aligned}
	\end{equation*}
	Где мы воспользовались формулой Стирлинга  и другими фактами:
	\begin{equation*}
		\begin{aligned}
			& N!=\sqrt{2 \pi N}(\frac{N}{e})^{N}                                \\
			& \overline{m} = Np                                                 \\
			& p = \frac{e^{-\frac{\Delta E}{kT}}}{e^{-\frac{\Delta E}{kT}} + 1} 
		\end{aligned}
	\end{equation*}
	Рассмотрим случаи:
	\begin{itemize}
		\item $T = 0 \qrq S = 0$
		\item $ T > 0 \qrq S>0$
		\item $T = \infty \qrq S = max$
		\item $T<0 \qrq S>0$, но это реализуемо только в двухуровневых системах
	\end{itemize}
	\subsection{Определение температуры через энтропию}
	Теперь можно дать определение температуры через энтропию. В районе средней энергии будет некий разброс. Система стремится повысить число способов и уменьшить энергию, но это процессы в неком смысле обратны друг дргу. Поэтому разброс, связанный с флуктуацией, будет очень небольшим. 
	В целом можно сказать, что $ln \Gamma = \alpha + \beta E + \gamma (E-\overline{E})^{2} + ...$. Вероятность тогда $p = e^{-\frac{E}{kT} + \alpha +\beta E + ...} $, и поскольку в самом вероятном состоянии линейных слагаемых по $E$ быть не должно, получаем $\beta = \frac{1}{kT}$. Температура теперь фактически является коэффициентом между логарифмом Г и энергией, то есть:
	\begin{equation*}
		\frac{\partial S }{\partial E} = \frac{1}{T}
	\end{equation*}
	Это назвается термодинамическим определением температуры. Так что изначальное определние температуры было не совсем верно. Данное же опредление применимо везде.  
	Для температуры важно тепловое равновесие. Для энтропии такого условия нет, её можно определить и в неравновесной системе. Например, можно разбить на равновесные подсистемы и сложить их энтропии. 
	\subsection{Начала термодинамики}
	Вспомним же уже приведённый пример из двух подсистем. Пусть у одной будет $T_{1}, E_{1}, S_{1}$, у второй $T_{2}, E_{2}, S_{2}$. Приведём их в соприкосновение (разрешим им взаимодействовать). Полная энергия, очевидным образом, из ЗСЭ просто суммируется. Что же произойдёт с полной энтропией системы? Она увеличится. Покажем это на двухуровневой системе. Пусть у одной все в основном состоянии, а у другой половина в возбужденном, половина нет. Тогда при сложении этих систем число способов вырастет и к тому же сильно (ведь там факториал числа
	частиц). Желающие могут найти подробное доказательство в 5 томе Теор. Физики ЛЛ. 
	
	Это и есть второе начало термодимики - энтропия возрастает. Третье начало термодинамики тоже было показано - при $T = 0$ энтропия тоже 0.
	
	Теперь можно сделать своеобразный "мостик" из стат. физики в термодинамику. При изменении температуры на $\Delta T$ произойдет следующее:
	\begin{equation*}	
		\Delta S = \frac{\partial S}{\partial T} \Delta T
	\end{equation*}
	Средняя энергия системы:
	$$
	\overline{E} = \frac{\sum \Gamma_{i} E_{i} e^{-\frac{E_{i}}{kT}}  }{\sum \Gamma_{i}  e^{-\frac{E_{i}}{kT}} }
	$$
	Внеся $\Gamma$ в показатель экспоненты, можно посмотреть на изменение средней энергии. Для удобства введем величину с названием статистическая сумма:
	\begin{equation*}
		\begin{aligned}
			& Z = \sum \Gamma_{i}  e^{-\frac{E_{i}}{kT}}                                                                                                                       \\
			& \frac{\partial Z}{\partial (\frac{1}{kT})} = - \sum E_{i}  e^{-\frac{E_{i}}{kT}+ ln \Gamma} = - \overline{E} Z                                                   \\
			& \overline{E} = \frac{\partial Z}{Z \partial (\frac{1}{kT}) } =  \frac{\partial (lnZ)}{ \partial (\frac{1}{kT}) } =  -\frac{\partial (lnZ)}{ \partial T } k T^{2} \\
			& \Delta E = T \frac{\partial S}{\partial T}                                                                                                                       
		\end{aligned}
	\end{equation*}
	\section{Лекция 5}
	Начнём с уже озвученных, но очень важных фактов:
	В замкнутой (теплоизолированной) системе энтропия не убывает.
	\begin{equation*}
		\begin{aligned}
			& T = \frac{\partial E}{\partial S} \\
			& F = E - TS (min)                  
		\end{aligned}
	\end{equation*}
	Теперь важно понимать, что энергия, как таковая, может быть определена с точностью до константы. Иными словами, к любой энергии мы можем прибавить или отнять константу. Но сейчас у нас нет понятия о "самой маленькой ступеньке". До этого мы определяли энтропию в дискретном понимании. Если мы разрежем пространство на очень много маленьких ступенек, мы получим одно значение энтропии. Разрежем на малое число ступеней - получим малую энтропию. Но так как мы вынуждены взять логарифм от числа способов, мы фактически просто добавляем константу к нашей энтропии. 
	Запишем стат. сумму и возьмём от неё логарифм. Основной вклад будут вносить состояния, находящиеся около среднего и тогда:
	\begin{equation*}
		TlnZ= Tln (\sum e^{\frac{kTln \Gamma_{n} - E_{n}} {kT}}) = -(E-TS)+C
	\end{equation*}
	\subsection{Хим. потенциал}
	У нас было распределение частиц по состояниям, и мы постоянно предполагали, что их число - константа. Теперь предположим, что их число может меняться. Примером реализации могут быть двумерные системы, скажем графен. И из металлического резервуара может прийти разное количество электронов. Но если меняется число частиц, меняется, очевидным образом, всё. Пусть у нас есть резервуар. В нём может быть N частиц, а может быть N+1 частиц. Понятно, что просто так в систему частицу не внести. Вероятность всё также пропорциональна Г на экспоненту, но и энтропия, и энергия будут функциями числа частиц. Введем величину с названием хим. потенциал. Она характеризует разность свободных энергий (с -) для случая с N и N+1 частицами.
	\begin{equation*}
		\begin{aligned}
			& \mu =- T (\frac{\partial S}{\partial N})                                                                            \\
			& p \propto e^{\frac{-\mu N + TS- E}{kT}} - \text{большое каноническое распределение} \\
		\end{aligned}
	\end{equation*}
	То есть фактически таким образом мы учитываем зависимость от числа частиц. Данный факт очень полезен при рассмотрении Ферми и Бозе, но в данном курсе они отсутствуют, так как относятся к квантовой статистике.
	\subsection{Процессы}
	Равновесная термодинамика хороша тем, что описывает состояние макропараметрами (температура, энтропия и так далее). Но если мы, условно, будем размешивать чай ложкой, процесс у нас не будет равновесный. Важными процессами являются квазистатические. Это такие, что в них система последовательно проходит равновесные состояния. 
	
	Представим газ под поршнем, на котором лежит куча песка. Не квазистатический процесс будет, если убрать рукой горсть песка. Поршень начнет колебаться, но мы не опишем это, поскольку нам попросту не хватит переменных. Локально можно говорить о равновесном процессе, но для всей системы так нельзя. Если же мы будем убирать по 1 песчинке, это будет примером квазистатического процесса.
	
	Процессы бывают обратимые и необратимые.
	\subsubsection{Адиабатический процесс}
	Рассмотрим замкнутую систему и будем в ней что-то менять. Энтропия не убывает, процесс обратимый и квазистатический. Тогда в соответствии с 2 началом термодинамики изменение энтропии равно нулю. Такой процесс называют адиабатическим.
	
	
	Полезным будет следующий факт: частная производная энергии по какому-то параметру, в предположении, что она от него зависит, равна: 
	\begin{equation*}
		(\frac{\partial \overline{E}}{\partial \lambda} )_{S}= \overline{(\frac{\partial E}{\partial \lambda})_{S}}
	\end{equation*}
	\subsection{Первое начало термодинамики}
	Дадим опрделение давлению и перепишем всё в терминах дифференциалов:
	\begin{equation*}
		\begin{aligned}
			& (\frac{\partial \overline{E}}{\partial V} )_{S}= p \\
			& dE (V, S) = TdS -pdV                               
		\end{aligned}
	\end{equation*}
	То есть определение энергии есть некая математика. Можно сказать, что первое начало термодинамики - закон сохранения энергии. Количество теплоты, подведенной к телу, расходуется на совершение телом работы и изменение его внутренней энергии. 
	Формулируется оно так:
	\begin{equation*}
		dQ=dE+pdV=TdS
	\end{equation*}
	Таким образом, от статистически определённой энтропии мы перешли к термодинамической энтропии. Она, как таковая, входит в самые разные тепловые характеристики тела.
	
	Вернёмся обратно к процессам. Вышеизложенный вывод относится только к квазистатическим процессам. Если наш чай в кружке как-то быстро вращается или что-то подобное, мы не можем описывать его полученными уравнениями. 
	\subsection{Циклы}
	Энергия - функция состояния системы. Она характеризует некоторое равновесное состояние системы. Температура, объём, давление, энтропия - тоже функции. Теплота же не является такого рода функцией. Для этого вспомним циклы. Это как раз о том, что теплота не является функцией состояния системы. Как известно, циклы обычно рисуются в pV координатах. Однако вне школьной термодинамики чаще пользуются TS координатами. Адиабатический процесс в данных координатах будет вертикальной прямой. Представим переход из точки A в точку B. Его можно реализовать всевозможными способами. Мы можем вычислить теплоту:
	\begin{equation*}
		Q = \int TdS	
	\end{equation*}
	Теплота - площадь под графиком и зависит от процесса, по которому оно подводится. Для каждого процесса можно ввести понятие теплоемкости. Раньше мы могли её вычислить, поскольку рассматривали на примере спинов, которые никуда не расширяются и т. д.  В идеальном газе это не так, к тому же, она характеризует именно процесс: 
	\begin{equation*}
		C = \frac{\partial Q}{\partial T}
	\end{equation*}
	
	Процесс можно сделать замкнутым, т. е. заставить систему перейти из одной точки в неё же. Под верхней кривой на диаграмме будет подведенное тепло, под нижней - отданное системой. Мы можем всё проинтегрировать и тогда, пользуясь нулевым изменением энергии (пришли ведь в ту же самую точку), получаем:
	\begin{equation*}
		dQ = A
	\end{equation*}
	\subsection{Идеальный газ}
	Уравнение состояния идеального газа:
	\begin{equation*}
		\begin{aligned}
			& p = nkT           \\
			& pV = \nu RT       \\
			& E = \frac{3kT}{2} 
		\end{aligned}
	\end{equation*}
	Для 1 моля (т. е. $\nu = 1$) вычислим энтропию:
	\begin{equation*}
		\begin{aligned}
			& T = (\frac{\partial E}{\partial S})_{V}  \\
			& p = -(\frac{\partial E}{\partial V})_{S} 
		\end{aligned}
	\end{equation*}
	Откуда:
	\begin{equation*}
		\frac{3R}{2} = (\frac{\partial S}{\partial lnT})
	\end{equation*}
	Подобным образом вычисляется и теплоёмкость:
	\begin{equation*}
		C = \frac{5}{2} R
	\end{equation*}
	\section{Лекция 6}
	Вспомним:
	\begin{equation*}
		T = (\frac{\partial E}{\partial S})_{V}
	\end{equation*}
	И 2 начало термодинамики, гласящее о том, что энтропия замкнутой системы не убывает.
	\subsection{Состояние равновесия}
	Возьмём два резервуара, один с $T_{1}$, другой с $T_{2}$ и больше ничего в нашей Вселенной нет. Объемы и полная энергия пусть будут константы: Тогда:
	\begin{equation*}
		\Delta S_{1} = \frac{\Delta E}{T_{1}}, \Delta S_{2} = - \frac{\Delta E}{T_{2}}
	\end{equation*}
	Если нарисовать гиперболу, станет очевидно, что $\Delta E > 0$ при $T_{2} > T_{1}$, или иными словами, тепло переходит от горячего к холодному. И естественно, что в состоянии равновесия температуры равны. 
	\begin{equation*}
		\begin{aligned}
			& Tds = dE+pdV                     \\
			& p =(\frac {TdS}{dV})_{E = const} 
		\end{aligned}
	\end{equation*}
	Откуда с аналогичными рассуждениями получается, что тело с меньшим давлением сжимается, а с большим - расширяется, и в равновесии они равны. То же самое и с хим. потенциалами, при равновесии они равны. 
	\subsection{Тепловые машины}
	Исторически многие учёные думали над тепловыми машинами. Это связано с развитием науки конца 18-19 веков. Одним из важнейших является цикл Карно, представляющий собой две изотермы и две адиабаты. В TS координатах он представляет собой прямоугольник. Очевидным образом, цикл Карно говорит о том, что машина работает с наибольшем КПД при заданных температурах (достаточно посмотреть на площадь внутри кривой и под кривой). Также, КПД зависит исключительно от температур, и не зависит от рабочего тела. 
	\begin{equation*}
		\text{КПД}=\eta=1 - \frac{T_{1}}{T_{2}}
	\end{equation*}
	
	Однако данный цикл можно прокручивать и в обратную сторону. Вместо работы от машины мы можем сами на неё воздействовать. Тогда мы запускаем цикл в обратную сторону и фактически передаём тепло от холодного тела горячему. Примером такой машины служит самый обыкновенный холодильник. Очевидно, "КПД" станет больше 1, хотя у тепловых машин он только меньше 1. 
	
	Тепловую машину по циклу Карно называют идеальной тепловой машиной, но ее реализовать не получается. Самым близким из имеющегося является турбина. Пример можно найти, посмотрев на турбины АЭС. У неё процесс, в котором совершается работа, очень близок к адиабате. В целом он представляет собой две почти адиабаты и две изобары. Чтобы КПД был высокий, нужно сильно нагревать. Лучшие турбины на сегодняшний день дают около 45\%.  
	\subsection{Термодинамические тождества}
	В предположении, что у нас идеальный газ и цикл Карно, можно доказать некоторые полезные утверждения.
	\begin{equation*}
		\begin{aligned}
			& \eta = \frac{A}{Q} = \frac{\Delta T}{T}                                                                        \\
			& A = \Delta p \Delta V = \Delta V \cdot (\frac{dP}{dT})_{V} \cdot  dT                                           \\
			& Q = (\frac{dE}{dV})_{T} \cdot dV+pdV                                                                           \\
			& \eta = \frac{dT}{T} = \frac{ dV \cdot dT \cdot (\frac{dp}{dT})_{V}  }{ dV \cdot ( p + (\frac{dE}{dV})_{T} )  } \\
			& p+(\frac{dE}{dV}) _{T}= T (\frac{dp}{dT})_{V}                                                                  
		\end{aligned}
	\end{equation*}
	Пользуясь тем, что энергия от объёма не зависит:
	\begin{equation*}
		p \propto T
	\end{equation*}
	\subsection{Процесс Джоуля-Томпсона}
	Рассмотрим движение газа через пористую перегородку. Направленного движения у нас нет. Пусть с разных концов перегородки разные давления, а сама она теплоизолирована. Можно написать для порции газа:
	\begin{equation*}
		U_{1} + p_{1} V_{1} = U_{2} + p_{2} V_{2} = I - \text{энтальпия}
	\end{equation*}
	Для идеального газа:
	\begin{equation*}
		\begin{aligned}
			& U = \frac{3pV}{2}                            \\
			& I = U + pV = \frac{5pV}{2} \propto T = const 
		\end{aligned}
	\end{equation*}
	Для идеального газа в данном процессе ничего существенного не произойдёт, поменяется объем и давление. Для того, чтобы что-то произошло, необходимо использовать неидеальный газ. 
	\section{Лекция 7}
	\subsection{Термодинамические потенциалы}
	\begin{itemize}
		\item $dU = TdS - pdV \qquad U \;	\text{-  внутренняя энергия}$
		\item $dF = -SdT - pdV \qquad F = U - TS \; \text{ - свободня энергия}$
		\item $dI = TdS + Vdp \qquad I = U +pV \;	\text{- энтальпия}$
		\item $dG = -SdT + Vdp \qquad G = U-TS+pV 	 \;\text{- потенциал Гиббса}$
	\end{itemize}
	Все эти величины - функции состояния системы, но в разных переменных. Их размерность равна размерности энергии. Уравнение состояния системы записывается, например, так: $U=U(S,V), F=F(T,V)$. Запишем определение температуры:
	\begin{equation*}
		T = (\frac{\partial U}{\partial S})_{V} = (\frac{dI}{dS})_{p}
	\end{equation*}
	И так можно расписать для всех величин. Продвинуться можно, написав соотношения Максвелла.
	\subsection{Соотношения Максвелла}
	\begin{equation*}
		dU = T(S,V)dS - p(S,V)dV
	\end{equation*}
	Взяв скрещенную производную, получаем:
	\begin{equation*}
		(\frac{\partial T}{\partial S})_{S} = -(\frac{dp}{dS})_{V}
	\end{equation*}
	И так можно сделать не только с данным уравнением:
	\begin{equation*}
		\begin{aligned}
			& (\frac{\partial S}{\partial V})_{T} = (\frac{dp}{dT})_{V}  \\
			& (\frac{\partial T}{\partial p})_{S} = (\frac{dV}{dS})_{p}  \\
			& -(\frac{\partial S}{\partial p})_{T} = (\frac{dV}{dT})_{p} 
		\end{aligned}
	\end{equation*}
	Таким образом, мы записали основные соотношения Максвелла. Конечно, их гораздо больше. Здесь, например, никак не учтён хим. потенциал. 
	\subsection{Потенциал Гиббса и уравнение Клайперона-Клазиуса}
	Данный потенциал прекрасен тем, что очень удобен для фазовых переходов. Для начала определим интенсивные и экстенсивные переменные.
	
	Интенсивные переменные - это те, которые описывают удельные величины и, можно сказать, не зависят прямо от объема системы. Их список: $T, p, \mu, B$. Если удвоить систему, с ними ничего не произойдёт. Экстенсивные - наоборот: $S,N,V,M$. Например, подсчитав стат. сумму мы можем подсчитать энергию, но не энтальпию. Именно поэтому нам и нужно столь много потенциалов и переменных.
	
	Как можно заметить, дифференциал п. Гиббса записывается как раз через интенсивные переменные. Рассмотрим равновесие жидкость-газ. В пересчёте на одну частицу у газа потенциал Гиббса меньше, чем у воды, поэтому происходил бы переход из жидкости в газ, обеспечивающий выигрыш в энергии. В случае равновесия перехода не происходит, поэтому потенциалы Гиббса двух фаз равны. Напишем уравнение кривой равновесия (p(T)) для одной частицы:
	\begin{equation*}
		\begin{aligned}
			& -S_{1} dT + V_{1}dp = -S_{2} dT + V_{2} dp                                                                         \\
			& (S_{1}-S_{2})dT=(V_{1}-V_{2})dp                                                                                    \\
			& \frac{dp}{dT} = \frac{S_{1}-S_{2}}{V_{1}-V{2}} \text{- Уравнение Клайперона-Клаузиуса} 
		\end{aligned}
	\end{equation*}
	\subsection{Фазовая диаграмма}
	Фазовая диаграмма всегда рисуется в интенсивных переменных, например, часто это делают в Tp координатах. На фазовой диаграмме при более высоких температурах и более маленьких давлениях - газ. То есть, если у нас жидкость-газ, жидкость будет над вышеупомянутой кривой. Мы сейчас говорим о переходах первого рода, которые имеют теплоту и здесь важен скачок энтропии. При малых температурах, как известно для воды, будет твёрдое тело. Очевидно, что в какой-то точке все три состояния пересекутся. Эту точку называют тройной точкой. Она есть не только у воды, а почти у всех материалов. Для воды она находится чуть теплее нуля по Цельсию. В воде очень важным моментом является наклон прямой жидкость-твердое тело. У неё коэффициент наклона отрицательный. Это связано с тем, что вода расширяется при замерзании.
	\begin{equation*}
		\frac{S_{hard}-S_{liquid}}{V_{hard}-V_{liquid}} < 0 
	\end{equation*}
	И лёд плавает. Это играет важнейшую роль для жизни на Земле, ведь если лёд тонул, холод бы шёл ко дну, и Земля быстро бы замёрзла. А так он плавает на поверхности и под ним живут рыбы и так далее. В метане, например, не так. 
	
	Теперь рассмотрим, что произойдёт при длительном нагреве воды. Тогда мы придём к точке, которую называют критической . Давление становится очень большим, молекулы очень часто в газе бьются друг о друга и у нас газ становится неотличим от жидкости. У воды она составляет около 218 атмосфер и 374 градуса Цельсия.
	
	У гелия-4 тройной точки нет. То есть на фазовой диаграмме будет только жидкость и газ. Тем не менее, критическая точка у него есть. Рассмотрим уравнение данной кривой:
	\begin{equation*}
		\begin{aligned}
			& p = nkT                                                                           \\
			& V = \frac{1}{n} = \frac{kT}{p} \quad V_{liquid} = const = \frac{m}{\rho_{liquid}} 
		\end{aligned}
	\end{equation*}
	Когда объём на одну молекулу газа много больше объёма на одну молекулу жидкости: 
	\begin{equation*}
		\begin{aligned}
			& \frac{dp}{dT} = \frac{qp}{T(kT)}, \; q  \; \text{- теплота плавления} \\
			& \frac{dp}{p} = \frac{q}{k} \cdot \frac{dT}{T^{2}}                                     \\
			& ln(\frac{p}{p_{0}}) = -\frac{q}{k}(-\frac{1}{T}+\frac{1}{T_0})                        \\
			& p = p_{0} e^{-\frac{q}{kT}}                                                           
		\end{aligned}
	\end{equation*}
	Где важно, что данная функция обязательно выходит из нуля. Чтобы получить из гелия-4 твёрдое тело, нужно очень маленькую температуру и давление порядка 25 атмосфер. В целом, данный газ известен тем, что в нём открыли сверхтекучесть. Это переход, проходящий в жидкой фазе при малом давлении и температуре. 
	
	Есть ещё гелий-3. Из-за отсутствия одного нейтрона на самом деле возникает очень большая разница. В нём кривая твёрдое тело-жидкость имеет сначала отрицательный наклон. Запишем: 
	\begin{equation*}
		\frac{dp}{dT} = \frac{S_{1}-S_{2}}{V_{1}-V{2}}
	\end{equation*}
	Но тут не как у воды. Дело в кристаллической решётке. Каждый элемент в ней имеет спин, который ориентирован, как угодно. На больших температурах они друг с другом никак не взаимодействуют. Происходит образование Ферми-жидкости, энтропия которой линейна по температуре.
	\begin{equation*}
		\frac{dp}{dT} = -\frac{Nln2 - \alpha T}{\Delta V}
	\end{equation*}
	Разница в том, что тут минус даёт числитель, а у воды - знаменатель. Гелий-3 тоже может быть сверхтекучий, но при температурах меньше 2 мК. Из интересных веществ можно привести в пример хром. У него нет жидкого состояния.
	\section{Лекция 8}
	\subsection{Газ Ван дер Ваальса}
	Для одного моля, запишем:
	\begin{equation*}
		pV=RT
	\end{equation*}
	Введём поправку на взаимодействие и минимальный объём, занимаемый молекулой. Взаимодействую молекулы посредством неких силовых полей. Например, обычно рассказывают про потенциал Леннарда-Джонса. Добавка пропорциональна квадрату концентрации:
	\begin{equation*}
		\begin{aligned}
			& (p+\frac{a}{V^{2}})(V-b) = RT                       \\
			& \frac{pV^{3}-pV^{2}b + aV - ab -RTV^{2}}{V^{2}} = 0 
		\end{aligned}
	\end{equation*}
	То есть перед нами на самом деле несколько усложнённое уравнение состояния. Относительно V мы имеем кубическое уравнение. Как это повлияет? В случае большой температуры получится следующее (пренебрегая малыми поправками):
	\begin{equation*}
		p = \frac{RT}{V-b}
	\end{equation*}
	\subsection{Критическая точка}
	Понижая температуру, у нас будет значению давления соответствовать больше решений, из-за чего, скажем, изотерма станет с некоторым "бугорком". В определенный момент возникнет точка и два горба сомкнутся. Первая производная (как и вторая) в такой точке будет 0. 
	\begin{equation*}
		\frac{\partial p}{\partial V} = 0 \qquad \frac{\partial^{2} p}{\partial V^{2}} = 0
	\end{equation*}
	Данную критическую точку нужно найти. Разложим вблизи крит. точки (выразив давление из первоначального уравнения):
	
	\begin{equation*}
		\begin{aligned}
			& p(V-V_{k})^3=0                                                       \\
			& (V-V_{k})^{3} = V^{3} - 3V^{2}V_{k} +3VV_{k}^{2} - V_{k}^{3}         \\
			& 3V_{k}^{2} = \frac{a}{p_{k}}                                         \\
			& 3V_{k} = b - \frac{RT}{p_{k}}                                        \\
			& V_{k}^{3} = \frac{ab}{p_{k}}                                         \\
			& V_{k} = 3b, \; p_{k} = \frac{a}{27b^{2}}, \; RT_{k} = \frac{8a}{27b} 
		\end{aligned}
	\end{equation*}
	
	В этой точке газ и жидкость неразличимы (на диаграмме жидкость снизу, газ сверху).  Рассмотрим изотерму, которая будет лежать, где жидкость. На одном из её участков:
	\begin{equation*}
		\frac{dp}{dV} > 0 
	\end{equation*}
	А такого быть не может. Почему? 
	\begin{equation*}	
		p = -(\frac{\partial F}{\partial V})_{T}
	\end{equation*}
	Но ведь F - минимально, а значит его вторые производные на самом деле >0. Продифференцируем обе части по объему и получим противоречие. Посмотрим внимательно на свободную энергию. Аналогичными рассуждениями становится ясно, почему у петли гистерезиса не может быть отрицательного наклона (поскольку в дифференциал свободной энергии пойдёт $MdH$).
	
	Тогда мы должны перескочить тот участок, чтобы не было противоречия. Но ведь так тоже не будет. Система разобьётся на две части, которые будут сосуществовать при одном давлении. Мы можем пройти сразу 2 путями. Но пути ведут в одно и то же состояние, а значит:
	\begin{equation*}
		S_{1} = S_{2} - \text{условие на фазовое равновесие при заданном давлении}
	\end{equation*}
	Состояния на запрещённых участках имеют названия: перегретая жидкость и переохлаждённый газ, в зависимости от той части, где они лежат. Перегретая жидкость - метастабильное состояние, демонстрацию которого легко найти.
	
	В состоянии перегретой жидкости у системы лишь локальный минимум. Она хочет попасть в состояние с абсолютным минимумом, но для того, чтобы ей туда попасть, ей нужно преодолеть барьер. Для воды, например, подойдёт брошенная песчинка, чтобы образовать пузырь и понизить свою энергию. Для переохлаждённого газа есть камера Вильсона.
	\subsection{Внутренняя энергия газа Ван дер Ваальса}
	\begin{equation*}
		\begin{aligned}
			& dE = TdS - pdV                          \\
			& T = const                               \\
			& dE = T \frac{dS}{dV}_{T} \cdot dV - pdV 
		\end{aligned}
	\end{equation*}
	Пользуясь соотношениями Максвелла:
	\begin{equation*}
		\begin{aligned}
			& \frac{dE}{dV} = T(\frac{dp}{dT})_{V} - p      \\
			& E = f(t) + \int (T(\frac{dp}{dT})_{V} - p) dV 
		\end{aligned}
	\end{equation*}
	Дифференцируя уравнение газа Ван дер Ваальса при постоянном давлении и подставляя:
	\begin{equation*}
		E = f(T) + \frac{a}{V}
	\end{equation*}
	Допустим газ очень разреженный, тогда его теплоёмкость совпадет с теплоёмкостью идеального газа. Получаем:
	\begin{equation*}
		f(T) = C_{V} T
	\end{equation*}
	\section{Лекция 9}
	\subsection{Поверхностные явления}
	\begin{equation*}
		\begin{aligned}
			& F = \sigma A, \; \text{где A - area, площадь} \\
			& dE = TdS - pdV+\sigma dA                                \\
			& F = E-TS                                                \\
			& dF = -SdT-pdV + \sigma dA                               \\
		\end{aligned}
	\end{equation*}
	Обозначим силу f и для единицы площади получим:
	\begin{equation*}
		f = - \sigma l dx
	\end{equation*}
	Отбросим зависимость от объёма и получим:
	\begin{equation*}
		\begin{aligned}
			& dF = -SdT + \sigma dA                                              \\
			& dE = TdS + \sigma dA                                               \\
			& S = -( \frac{\partial F}{\partial T})_{A} = - \frac{Ad \sigma}{dT} \\
			& E = A(\sigma - T \frac{d \sigma}{dT} )                             \\
		\end{aligned}
	\end{equation*}
	Какова физическая причина того, что $\sigma > 0$ и почему система хочет минимизировать свою площадь? Рассмотрим поверхность жидкости. На каждую молекулу действуют её соседи и каждый её тянет в свою сторону. В среднем сила равна нулю. Но есть молекулы, находящиеся на краю, на который соседи действуют только с одной стороны. Образовывать "бугорок" таким молекулам невыгодно, поэтому $\sigma > 0$. 
	
	Как таковой, коэффициент $\sigma$ характеризует поверхность, или "интерфейс". Больше нуля он для интерфейса жидкость-газ. Так, например, ртуть собирается в капельки, если разбить градусник. Для твердого тела-жидкость это уже не так. Возможны оба случая, и положительный, и отрицательный. При отрицательном коэффициенте говорят, что происходит смачивание. Например, так вода растекается в лужицу на поверхности обезжиренного стекла. 
	
	Рассмотрим границу 3 сред (стол, на котором лежит капля). Проведём касательную в месте, где капля соприкасается со столом и рассмотри маленький элемент длины. Со стороны твердого тела каплю будет распирать (считаем, что происходит смачивание), а со стороны газа сжимает. Можно провести более интересный опыт: льём масло на воду и добавляем каплю эфира такую, что она находится аккурат посередине жидкостей. Исходя из точек касания можно вычислить все углы, зная коэффициенты поверхностного натяжения. 
	
	Коэффициент поверхностного натяжения уменьшается с ростом температуры и около критической точки стремится к 0.
	\subsection{Поднятие жидкости в капилляре}
	Пусть у нас есть резервуар и капилляр из смачиваемого материала. Тогда в капилляре мениск будет поднят. Косинус угла смачивания определяется аналогично предыдущим идеям. Как определить высоту подъёма? Нужно записать полную энергию системы и найти минимум (всё сводится к механической энергии). Пусть площадь капилляра $\pi r^{2}$, высота ц. м. $0.5h$, энергия:
	\begin{equation*}
		\begin{aligned}
			& E = \frac{ \pi r^{2} \rho h^{2} g}{2} - 2 \pi r h \sigma \\
			& \frac{\partial E}{\partial h} = 0                        \\
			& h =\frac {2 \sigma}{\rho g r}                            
		\end{aligned}
	\end{equation*}
	Есть вещетсва, которые сами идут к поверхности. Примером будет обычное мыло. 
	\subsection{Лапласово давление}
	Теперь вспомним, что у поверхности есть два радиуса кривизны. Возьмём цилиндр и посмотрим на силы, действующие на маленькую его полоску, после чего вычислим:
	\begin{equation*}
		\begin{aligned}
			& F_{sum} = \sigma L \phi                           \\
			& p =\frac{ F_{sum} }{ L \phi r} = \frac{\sigma}{r} 
		\end{aligned}
	\end{equation*}
	Если же радиуса два, их нужно складывать, т. е.:
	\begin{equation*}
		p =\sigma( \frac{1}{r_{1}}+ \frac{1}{r_{2}})
	\end{equation*}
	\subsection{Мыльные пузыри}
	Давление Лапласа внутри пузыря будет равно:
	\begin{equation*}
		p = \frac{4 \sigma}{r}
	\end{equation*}
	Потому что у пузыря две поверхности (как бы внутри и снаружи). То есть давление внутри пузыря избыточно. 
	\subsection{Тепловое расширение}
	Любое вещество расширяется под действием температуры. Обозначим его так:
	\begin{equation*}
		\begin{aligned}
			& \alpha =( \frac{\partial V}{\partial T} )_{p}\frac{1}{V} \\
			& pV = RT                                                  
		\end{aligned}
	\end{equation*}
	Газы стремятся расширяться, при фиксированном давлении их объём увеличивается. В твёрдом теле атомы находятся каждый в потенциальной яме. Они испытывают колебания. Полная энергия равна $kT$. С ростом температуры среднее положение атома смещается относительно самой нижней точки ямы. То есть ключевую роль в данном явлении играет ангармонизм. Всё тело начинает расширяться. Как оказывается, тепловое сужение тоже существует. Воспользовавшись соотношениями максвелла, получаем:
	\begin{equation*}
		\frac{\partial V}{\partial T} = - \frac{\partial S}{\partial p}
	\end{equation*}
	Например резина, когда её изотермически растягивают, её энтропия падает, поскольку хаотичные волокна начинают упорядочиваться. Это говорит о том, что у неё отрицательный коэффициент теплового расширения.
	\section{Лекция 10}
	\subsection{Фазовые переходы 2 рода}
	Учтем поправку от электрического поля в свободной энергии (поправка вытекает из курса электричества и магнетизма в прошлом семестре), а магнитное поле аналогично электрическому:
	\begin{equation*}
		dF = -SdT - pdV + \int_{V}  \frac{EdD+HdB}{4 \pi} dV
	\end{equation*}
	Здесь и далее мы будем рассматривать магнитные вещества. Безусловно, существуют оба вида, но в жизни нам ближе магнитные, а электрические будут абсолютно аналогичны. Представим себе газ из магнитных моментов. Магнитный момент, кстати говоря, у электрона больше, чем у ядра. Механический момент равен постоянной Планка для любого из случаев. Забудем про отличия СГС и СИ и запишем:
	\begin{equation*}
		\mu = evr
	\end{equation*}
	Приложим поле $H$:
	\begin{equation*}
		\begin{aligned}
			& B = 4 \pi M + H                                         \\
			& \overline{\mu} = \mu_0 \cdot tanh(\frac{\mu_{0} H}{kT}) \\
			& M = n \overline{\mu}                                    
		\end{aligned}
	\end{equation*}
	Получаем намагниченности магнетика, состоящего из одиночных спинов.
	
	Ещё некоторые могут помнить, что существует такая величина, как магнитная восприимчивость:
	\begin{equation*}
		(\frac{dM}{dH} )_{H=0}= \frac{n \mu_{0}^{2}} {kT}
	\end{equation*}
	Данный закон имеет название закона Кюри. Вещества, которые ему подчиняются и у которых магнитная восприимчивость положительна, называются парамагнетики. Причём нужно заметить, что магнетизм - квантовое явление (следует из того, что $\mu_{0}$ пропорционально постоянной Планка)
	
	Ещё бывают ферромагнетики. Любое вещество при высоких температурах в лучшем случае будет парамагнетиком и его магнитная восприимчивость мала. Как же заставить парамагнетик намагнититься? Когда спины были свободны, было удобно считать, так как они не взаимодействовали. Тут же ситуация другая - никакой фазовый переход без взаимодействия не получить. Мы должны "включить" взаимодействие. 
	\subsection{Теория среднего поля}
	\begin{equation*}
		E_{spin} = i \mu (H + \alpha M)
	\end{equation*}
	Где $\alpha$ - константа Вейса. Теперь подсчитаем среднее значение спина. 
	\begin{equation*}
		\begin{aligned}
			& E_{up} = - \mu (H + \alpha M)                                                                                                                                                          \\
			& E_{down} = \mu (H + \alpha M)                                                                                                                                                          \\
			& M = n \mu =\frac{ n \mu_{0} \cdot (  e^{\frac{\mu (H+ \alpha M)}{kT}}  -  e^{\frac{-\mu (H+ \alpha M)}{kT}})} {e^{\frac{\mu (H+ \alpha M)}{kT}}  +  e^{\frac{-\mu (H+ \alpha M)}{kT}}} 
		\end{aligned}
	\end{equation*}
	Положим $H=0$ и $M$ - малой, решим относительно $\mu$:
	\begin{equation*}
		M = n \mu_{0} \frac{2 \mu_{0} \alpha M}{kT(2+\frac{ \mu_{0} ^{2} \alpha^2 M^2}{(kT)^2}) }
	\end{equation*}
	У нас имеется два решения:
	\begin{itemize}
		\item $M = 0$ - высокие температуры
		\item $M \neq 0$ - низкие температуры
	\end{itemize}
	Рассмотрим случай маленького поля, тогда $M \propto H$ - решим линейного отклика. Тогда $M = \chi H$. 
	\begin{equation*}
		\begin{aligned}
			& \chi H = 2n \mu_{0}^{2} ( (1+\alpha \chi)\frac{H}{kT}) \\
			& \chi = \frac{\chi_{0}} {T-T_{c}}                       
		\end{aligned}
	\end{equation*}
	Или закон Кюри-Вейса.
	Ферромагнетик находится ниже температуры $T_{c}]$.
	\subsection{Теория Ландау}
	Фазовый переход 2 рода отличается от фазового перехода жидкость-газ. Чем ниже критической температуры, тем выше спонтанная намагниченность, а при крит. температуре она нулевая. Нужно разложить вблизи точки перехода свободную энергию так, чтобы образование спонтанного магнитного момента из неё следовало. Поскольку $M$ и $-M$ равновероятны, наша свободная энергия должна раскладываться только по чётным степеням.
	\begin{equation*}
		F = F_{0} + \alpha M^{2} + \beta M^{4}
	\end{equation*}
	Логично, что коэффициенты должны как-то зависеть от температуры. $\alpha$, например, не может быть всегда положительной, ведь при малых магнитных моментах вклад от $M^{4}$ попросту исчезнет, а нас интересуют ненулевые решения.  $\beta$ при это может быть всегда положительной.
	\begin{equation*}
		F = F_{0} + a (T_{c} - T) M^{2} + \beta M^{4}
	\end{equation*}
	Теперь поисследуем данную функцию. Запишем минимум свободной энергии и найдём, как магнитный момент зависит от температуры:
	\begin{equation*}
		\begin{aligned}
			& \frac{\partial F}{\partial M} = 0 \\
			& 2Ma(T-T_{c}) + 4 \beta M^{3} = 0  
		\end{aligned}
	\end{equation*}
	Помимо очевидного нулевого решения:
	\begin{equation*}
		MT = \pm \sqrt{\frac{a(T_{c} - T)}{2 \beta}}	
	\end{equation*}
	Ключевым моментом в теории Ландау является тот факт, что он предположил существование $M$ ниже критической температуры. $M$ называют параметром порядка. Но это не единственный пример фазового перехода второго рода. Ещё одним примером может быть критическая точка газа Ван дер Ваальса, который описан в 5 томе теор. физики ЛЛ. Ещё пример - сверхпроводник.
	\section{Лекция 11}
	\subsection{Модель Изинга}
	Жизнь устроена сложнее, чем было показано в прошлый раз. Вспомним: материал в соответствии с теорией среднего поля должен описываться средним значением. На самом деле вблизи $T_{c}$ у нас $M$ очень мало. А как мы помним из флуктуаций, чем меньше величина, тем больше её относительные флуктуации.
	\begin{equation*}
		\frac{\delta m}{m} \propto \frac{1}{\sqrt{m}}
	\end{equation*}
	Тогда $M$ - некая функция, и нужно перейти к функциональному интегралу, что будет достаточно сложно. Делать мы этого в данном курсе не будет. Вблизи любого переход есть флуктуационная область. Очень важным оказывается размерность системы. Теория среднего поля работает при размерности системы больше 2. Иначе она не работает. Это приводит к тому, что фазового перехода как такового нет, он разрушен флуктуациями. Существует иллюстративная модель. Модель Изинга. Рисуют её как систему спинов. Расставить их можно как мы хотим. Магнитное поле делается равным 0 и мы рассматриваем спин. Но все фазовые переходы возникают только при наличии взаимодействия. В каждой точке есть спин, который может направлен либо вверх, либо вниз. Пусть для упрощения у нас значения принимают либо 1, либо -1.  Энергия:
	\begin{equation*}
		E = - J \sum_{connections} s_{i} s_{j}
	\end{equation*}
	$J$ - константа взаимодействия. Её знак может быть любым. $s_{i}$ - значение некого спина. Второе значение - его соседа. Отсюда очевидно, что спинам выгодно смотреть в одну сторону, ведь энергия тогда понижается. Если температура равна нулю, все спины повёрнуты в одну сторону. Но у нас она не ноль, поэтому минимизируется свободная энергия, в которой возникает энтропия.
	
	Можно промоделировать методом Монте-Карло:
	\begin{equation*}
		\begin{aligned}
			& F = - T lnZ                        \\
			& Z = \sum_{i} e^{-\frac{E_{i}}{kT}} 
		\end{aligned}
	\end{equation*}
	Если решётка маленькая, то перебрать все состояния возможно. 1D считается аналитически. В 2D уже не факт, а 3D точно не считается. Метод же заключается в том, что вместо взятия всех состояний, мы берём лишь большое количество случайных. Так мы можем подсчитать стат. сумму при одной температуре, потом при другой и вычислить, к примеру, теплоёмкость.
	
	Почему в одномерной системе нет фазового перехода? Пусть температура не нулевая, но очень низкая. Температура не ноль, энтропия тоже. У нас есть бесконечная цепочка спинов. Пусть до определенного момента они все были повернуты вверх, а после него - вниз. Посчитаем энергию. Она имеет почти минимум (лишь один член войдёт с положительным знаком), но и его хватит, чтобы $M = 0$.
	\subsection{Основы кинетики}
	До этого мы рассматривали равновесные состояния. Сейчас же мы будем рассматривать изменения в системе, поскольку переходим к основам кинетики.	Кинетика же подразумевает обратное. Здесь мы будем что-то фиксировать из интенсивных параметров, а другие - менять. То есть у нас есть такая вещь, как поток тепла от горячего тела к холодному. Исходя из логики:
	\begin{equation*}
		q \propto - \alpha \Delta T
	\end{equation*}
	Можно, конечно, допустить и кубическую зависимость (но не квадратичную, нам важно, что тепло идет именно от горячего к холодному), но такое встречается довольно редко.
	\subsection{Уравнение диффузии и уравнение теплопроводности}
	Есть похожее явление - диффузия. Тут уже отличается концентрация. Думаем про 3D пространство (на примере диффузии). нарежем пространство на «слайсы», и пусть концентрация зависит от координаты. Скажем, это трубка, заполненная смесью газов. Ветер не дует, но молекулы двигаются. В каждом отдельном слое у нас локальное равновесие.
	\begin{equation*}
		\begin{aligned}
			& j = \frac{n(x)-n(x+dx)}{dx} \cdot D = - D \frac{dn}{dx} \\
			& I = Sj                                                  \\
			& \overline{j} = - D \overline{\nabla} n                  \\
		\end{aligned}
	\end{equation*}
	Концентрация частиц - плавная функция. Посчитаем два потока для одного из слоёв:
	\begin{equation*}
		\frac{\partial n}{\partial t} =\frac{( j_{left} - j_{right})S} {S dx} = \frac{-D \frac{\partial n}{\partial x} (x) + D \frac{\partial n}{\partial x} (x+dx)}{\partial x} = D \frac{\partial^{2} n}{\partial x^2}
	\end{equation*}
	Аналогично можно получить и результат для 3D:
	\begin{equation*}
		\frac{\partial n}{\partial t} = D \Delta n 
	\end{equation*}
	Проведя вывод, абсолютно аналогичный предыдущему, получим для теплопроводности:
	\begin{equation*}
		\begin{aligned}
			& \overline{q} = - \kappa \nabla \overline{T}               \\
			& \frac{\partial T}{\partial t} = \frac{\kappa}{C} \Delta T 
		\end{aligned}
	\end{equation*}
	Решать данные уравнения достаточно трудно, поэтому рассмотрим частные случаи. Например, у нас есть трубка длины $L$, на одном конце которой запаяно и налита вода. В комнате влажность равна нулю. Над самой водой влажность, очевидно, максимальна. Пусть у нас изотермический процесс. Предположим также, что у нас поток постоянный.
	\begin{equation*}
		\begin{aligned}
			& j = const = -D \frac{dn}{dx} = - \frac{n_{max}}{L} \\
			& I = DS\frac{n_{max}}{L}                            
		\end{aligned}
	\end{equation*}
	По понятным причинам нас интересуют $\kappa, D$ и было бы неплохо их найти. Как это сделать? В кинетике у нас появилась переменная времени. Раньше этого у нас не было. Возьмём такой процесс, который мы в силах посчитать (по понятным причинам, круг таких весьма ограничен).
	\subsection{Длина свободного пробега}
	Слукавим и забудем, что в идеальном газе частицы не взаимодействуют. Если у нас в газе молекула летит, она заметает собой некоторую область (цилиндр). Если в этой области будет другая частица, случится рассеяние. Воспользуемся тем, что в идеальном газе всё случайно и попробуем найти расстояние, которое проходит одна частица, не сталкиваясь с другими. Назовём его длиной свободного пробега. При пролёте некоего расстояния $dx$ есть вероятность, равная $\frac{dx}{L}$, столкнуться с другой частицей. $L $- та самая длина св. пробега. 
	
	Поскольку молекулы летают хаотически, поэтому для простоты предлагается их заморозить. Если они летают, они каждый раз будут оказываться на случайных местах. Ответ это не изменит, ведь у вероятности никакой корреляции в системе нет. Теперь об оценке. Пусть у нас есть сечение рассеяния $\sigma$ и все остальные молекулы распределены случайно, мы можем говорить о доле объёма пространства, в которую нужно попасть для свершения рассеяния. Она равна $n \sigma L$. Если она равна 1, значит мы гарантированно столкнулись.
	\begin{equation*}
		L = \frac{1}{\sigma n}
	\end{equation*}
	Стоит отметить, что $\sigma$ очень мала. 
	
	Теперь проведём такой мысленный эксперимент. Направим "покрашенные" частицы в идеальный газ. Их импульс будет падать до нуля. Из вероятности рассеяния напрямую следует:
	\begin{equation*}
		j = j_{0} e^{-\frac{x}{L}}
	\end{equation*}
	\subsection{Оценка коэффициента диффузии}
	У нас есть частица, у неё есть длина свободного пробега. Рассмотрим одномерный случай. Пусть у частицы есть скорость $v$, тогда $\tau_{run} = \frac{L}{v}$. Поток частиц вправо и влево:
	\begin{equation*}
		vne^{-\frac{x}{L}} ,\; v(n+dn)e^{-\frac{x}{L}}
	\end{equation*}
	Эти два потока мы можем вычитать лишь в очень узком диапазоне. С учётом экспоненциального затухания данный диапазон оценивается как $dL$, тогда:
	\begin{equation*}
		D = vL
	\end{equation*}
	Это в одномерном случае. В 3D:
	\begin{equation*}
		D = \frac{vL}{3}
	\end{equation*}
	Что очевидно следует из того, как меняется длина свободного пробега и скорость.
	\section{Лекция 12}
	\subsection{Вязкость}
	Вспомним механику 1 курса. Если у нас есть две пластины, верхняя движется со скоростью $v$, нижняя стоит на месте. Посередине вода. Вязкость характеризует связь между касательным напряжением трения слоёв друг о друга и градиентом скорости.
	\begin{equation*}
		\eta \frac{\partial v_{x}}{\partial z} = \tau_{x} = \frac{\Delta P_{x}}{\Delta t S}
	\end{equation*}
	Где $\tau_{x}$ - напряжение. Посчитаем импульс вдоль оси $x$, который уносят за собой частицы, движущиеся слева направо. Он равен:
	\begin{equation*}
		Sm\lambda \frac{n}{2} \frac{dv_{x}}{dz} 
	\end{equation*}
	Тогда можно выразить коэффициент, плюс сразу же проведём аналогию для уравнения теплопроводности:
	\begin{equation*}
		\begin{aligned}
			& \eta = \frac{1}{3} m v \lambda n         \\
			& \kappa =\frac{1}{3} \rho C_{V} v \lambda \\
		\end{aligned}
	\end{equation*}
	Чем больше молекул в газе, тем, казалось бы, более вязкий газ. А значит с ростом концентрации должна расти и вязкость. Однако в $\lambda$ (длина свободного пробега) содержит концентрацию в знаменателе. И зависимость от концентрации пропадает. Экспериментально это подтверждается. 
	
	\subsection{Теплопроводность газа}
	На похожем принципе работает датчик для измерения теплопроводности газа. Используется спираль нагревателя на тонких проволочках и термопара. Поскольку при нагревании газ начнёт отводить тепло, можно регистрировать изменения на термопаре. Называется такой датчик ПМТ-2. Ещё бывают датчики Пирани. С каждым видом датчиков можно поработать на наших лабораторных работах. Когда газ становится разреженным, у нас меняется его режим теплопроводности. Если раньше мы могли проводить аналогии с жидкостью, в данной ситуации так нельзя. Рассмотрим случай, когда молекула совсем одна и сталкивается лишь со стенками сосуда. И коэффициент теплопроводности, и вязкость становятся пропорциональны концентрации. Такой режим называется баллистическим. Особенностью гидродинамического режима является тот факт, что трения покоя в воде нет.
	
	Рассмотрим кусок металла с большим количество примесей. Электроны на них рассеиваются. В самых чистых системах можно добиться длины свободного пробега электрона в десятки микрометров. Данный режим называется диффузионным. Электрон теряет свой импульс полностью на примеси. В жидкости же импульс не релаксирует.
	
	\subsection{Броуновское движение}
	Пусть есть некая частица, например пыльца. У неё есть средняя кинетическая энергия и по теореме о равнораспределении по степеням свободы у неё равна $1.5 kT$, откуда её скорость:
	\begin{equation*}
		v = \sqrt{\frac{3kT}{m}}
	\end{equation*}
	Частица содержит миллиарды атомов, но корень сильно уменьшает это количество. Скорость получается порядка сантиметра в секунду. Для частицы можно нарисовать её диффузионную (её фрактальная размерность будет равна 2). Что же такое диффузия? Это случайное движение. Любая молекула в газе, в том же воздухе, пусть будет как-то помеченная, возможна для наблюдения. Молекула будет испытывать диффузию. Из выведенного ранее:
	\begin{equation*}
		D = \frac{1}{3} v \lambda
	\end{equation*}
	Пусть в каждый момент времени частичка делает шаг в случайном направлении. Время, которое тратится на один такой шаг:
	\begin{equation*}
		\Delta t = \frac{\lambda}{v}
	\end{equation*}
	Для упрощения перейдём в одномерие. Шаги можно делать влево и вправо. Подсчитаем среднее смещение. Очевидным образом, ответ 0. Подсчитаем средний квадрат. Вспомнив прошлые лекции, получим:
	\begin{equation*}
		\overline{\Delta x^{2}} = \frac{L \lambda^{2}}{\Delta t} = tD
	\end{equation*}
	Данное заключение обобщается и на случай большей размерности. Любая частица может быть помещена в силовое поле. Есть опыт Милликена. Берут заряд, помещают его на частицу. Полученную конструкцию помещают в конденсатор. Можно добиться того, чтобы сила тяжести была скомпенсирована. Так впервые измерили заряд электрона.
	\subsection{Трение}
	Если мы приложим к частичке некую силу, она приобретёт некую установившуюся скорость. Сила трения:
	\begin{equation*}
		F = 6 \pi \eta r v - \:	 \text{формула Стокса}
	\end{equation*}
	Но даже к направленному движению добавляется случайное. Броуновская частица, на самом деле, очень сильно трётся о своих соседей. Коэффициент, связывающий силу и скорость называется подвижностью. Обозначается он $B$. Можно ли получить какое-то соотношение между коэффициентом диффузии и подвижностью? Да, существует формула Эйнштейна-Смолуховского. Пусть у нас термодинамическое равновесие, сила направлена вниз (ось направлена вверх) и постоянна. Есть потенциальная энергия $Fx$. Частицы будут распределены по Больцману. Концентрация наших частиц $n = n_{0} e^{-\frac{Fx}{kT}}$. В состоянии равновесия наверх будет действовать диффузия.
	\begin{equation*}
		j = D \frac{\partial}{\partial x} (n_{0} e^{-\frac{Fx}{kT}})
	\end{equation*}
	С другой стороны, на верхние частицы действует сила и они стремятся вниз. С усётом равенства потоков
	\begin{equation*}
		FB n_{0} e^{-\frac{Fx}{kT}} = \frac{D F n_{0} e^{-\frac{Fx}{kT}}} {kT}
	\end{equation*}
	\begin{equation*}
		D = kTB
	\end{equation*}
	\subsection{Блуждания и человек}
	Броуновская частица будет блуждать. Её отдаление мы уже подсчитали. Теперь включим малое поле. Тогда она будет смещаться и характерное смещение будет стремиться на больших временах к $BFt$. Частично это можно переложить и на человека: человек движется в пространстве своих возможностей. Если он делает это случайно, его "смещение" пропорционально корню из времени. Коэффициенту $D$ можно сопоставить способности человека. То есть такой человек успеет уйти далеко сразу во всём. Если же на человека действует мотивация, она же "сила", его движение будет пропорционально уже времени. И такой человек уйдёт гораздо дальше. Именно поэтому у человека должна быть цель. А роль внешних факторов можно сопоставить температуре. 
	\subsection{Волна}
	Волна в линейном понимании - решение волнового уравнения. 
	\begin{equation*}
		\Delta p = \frac{1}{c^{2}} \frac{\partial^{2} p}{\partial t^{2}}
	\end{equation*}
	Волны бывают разные. Ранее мы уже видели электромагнитные в прошлом семестре. Ещё знакома волна в струне. Из интересных бывают капиллярные волны, то есть волны на поверхности воды. Есть капиллярно-гравитационные.
	\begin{equation*}
		c = \sqrt{\gamma \frac{p}{\rho}}
	\end{equation*}
	Это скорость звука в газе, в металле же она иная, $E$ - модуль Юнга:
	\begin{equation*}
		c = \sqrt{\frac{E}{\rho}}
	\end{equation*}
	\section{Лекция 13}
	Данная часть выходит за рамки обычного курса и предназначена, чтобы в очередной раз показать, что термодинамика не ограничена идеальным газом.
	\subsection{Термодинамика электронов в металле}
	Строгий вывод на данном этапе дать не получится, поскольку он требует знаний в квантовой механике. В металле есть кристаллическая решётка. Металлическая связь реализуется электронами, которые "летают" из атомов по всей решётке. В щелочных металлах типична ситуация, когда каждый атом даёт один электрон. Атомов примерно $10^{2}$ на каждый $\text{см}^{3}$. В идеальном газе же (например, скажем что воздух в комнате - идеальный) оценим число так: у воды плотность тонна на кубометр. У воздуха она примерно в 1000 раз меньше. С учётом того, что металл обычно плотнее воды, получим оценку порядка $10^{18}-10^{19}$ атомов на $\text{см}^{3}$. То есть идеальный газ действительно идеальный в том смысле, что атомы летают независимо друг от друга. Электроны в металле - плотная материя. В таком случае электроны часто летают мимо друг друга и занимать одни и те же места. Это, вероятно, напомнит про электроны в атомах. Как известно, существует несколько уровней и заполняются они снизу-вверх. Существует запрет Паули, гласящий о том, что два фермиона (частицы с полуцелым спином) не могут находиться в одном состоянии. То есть на $1S$ орбиталь никак не может попасть третий электрон. В металле также число мест ограничено. Вспомним теперь распределение Гиббса, которое для невзаимодействующих частиц свелось к распределению Максвелла:
	\begin{equation*}
		p \propto e^{-\frac{E}{kT}}
	\end{equation*}
	Когда материя не плотная, как тот же идеальный газ, на любую позицию претендует только одна частица. Для фермионов распределение иное:
	\begin{equation*}
		p = \frac{1}{e^{\frac{E-\mu}{kT}} + 1}
	\end{equation*}
	Оно называется распределением Ферми-Дирака. Очень существенно, что для данного распределения принцип Паули выполняется. Именно так и устроены металлы.  Когда температура близка к 0, распределение электронов по энергиям становится очень резким. $\mu$ - их характерная кинетическая энергия, порядка электронвольта. Принципиальная разница с идеальным газом заключается в том, что электроны вынуждены летать с большими энергии при малых температурах, вместо того чтобы останавливаться. 
	\subsection{Собственный хим. потенциал}
	Чем ещё принципиально отличаются электроны от молекул идеального газа? Электроны имеют заряд. Вспомним определение хим. потенциала:
	\begin{equation*}
		\mu = \frac{\partial F}{\partial N}
	\end{equation*}
	Когда мы начинаем думать про электроны в металле возникает следующая проблема. Допустим мы изменили число частиц на один. Понятно, что из вакуума электрон взяться не мог. То есть электрон должен был где-то на бесконечности существовать. Вопрос, который возникает сразу же: а что такое бесконечность и где она? Введём некий уровень энергии электрона и назовём его уровнем на бесконечности (например, сгодится просто другой массивный кусок металла). Но у электрона есть заряд, поэтому когда мы его внесём в металл, у нас возникнет очень малый отрицательный заряд. У металла появляется потенциал, что приводит к следующему: при внесении последующих электронов придётся совершать ещё и работу против заряженного куска металла. Тогда работа по переносу из бесконечности должна это учитывать: $\mu + e \phi$. То есть помимо собственного хим. потенциала, у металла будет и электрический потенциал. Собственный потенциал называют работой выхода. Именно он и встретится в будущем при фотоэффекте в уравнении Эйнштейна. Вся же сумма имеет название электрохим. потенциала. И именно её измеряет вольтметр. То есть на самом деле, батарейкой мы задаём разность электрохим. потенциалов. Несколько простых примеров:
	
	Пример, подобный эксперименту лорда Кельвина. Был взят конденсатор, у которого одна обкладка из цинка, а другая из меди и амперметр, подключенный к ним. У металлов разные хим. потенциалы, но в силу нулевой разности электрохим. потенциалов (ведь их соединили последовательно через провод и амперметр) на одну сторону натекли +, на другую -. Получилось следующее:
	\begin{equation*}
		\mu_{Zn} - \mu_{Cu} = e \phi = eEd
	\end{equation*}
	Ток через амперметр равен нулю. Теперь, если потрясти конденсатор, изменится расстояние между обкладками. Разность работ выхода останется постоянной, но в силу изменения расстояния $d$ будет изменяться заряд и эл. поле. А следовательно, потечёт ток, что позволяет измерить разность собственных потенциалов. Данный метод используется и по сей день.
	\subsubsection{Эффект Зеебека и Пельтье}	
	Существует термоэлектрический эффект Зеебека, суть которого в том, что хим. потенциал зависит от температуры. На этом принципе и работает термопара. Когда температура у пары металлов на концах разная, возникает ненулевая разность потенциалов.
	
	Существует и обратный эффект, названный в честь Пельтье. На их основе делают мини холодильники. Основным материалом для элементов, работающих на данном эффекте, являются полупроводники.
	
	Откуда берутся эти эффекты? Посмотрим на распределение Ферми. "Горячие" электроны более податливы и подвижны. Они и уйдут первыми, поэтому система будет термализоваться с более низкой температурой.
	%таймкод 33:41
\end{document}
